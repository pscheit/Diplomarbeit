\chapter*{Abstract}

Die objektorientierte Programmierung ist in der heutigen Softwareentwicklung nicht mehr wegzudenken. Auf Sprachelemente wie Kapselung, Polymorphie und Vererbung kann in modernem Softwaredesign nur schwer verzichtet werden. Ebenso sind relationale Datenbanken noch immer die erste Wahl für die persistente Speicherung von Daten. Sie skalieren für große Datenmengen, sind sicher, verwalten Transaktionen zwischen Benutzern und besitzen den größten Marktanteil von allen Datenbanken. \\
Die Sammlung der Probleme die auftreten, wenn man objektorientierte Datenstrukturen in einer relationalen Datenbank speichern möchte, werden als \term{Object-relational Impedance Mismatch} bezeichnet. \\
In dieser Arbeit werden die entstehenden Probleme, die durch den Unterschied zwischen den beiden Paradigmen enstehen, erläutert und bisher gefundene Lösungen für den \term{Impedance Mismatch} vorgestellt. Es wird gezeigt, warum es keine einzige perfekte Lösung für den \IM geben kann, weshalb die bisher in der Industrie entwickelten Persistenzmechanismen in vielen individuellen Bereichen Anwendung finden. \\
Die Strategien werden kritisch betrachtet und eigene Verbesserungen und Konzepte ausgearbeitet, die eine Basis für ein weiteres Framework für Webapplikationen legen. Die Effizienz dieser verbesserten Konzepte wird durch einen Vergleich mit ähnlichen Produkten untersucht.
